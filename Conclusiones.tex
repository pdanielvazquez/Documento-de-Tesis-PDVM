\renewcommand{\chaptername}{Capitulo}
\chapter{Conclusiones} 
\section{Conclusiones}

A trav\'es del servicio web desarrollado se ha establecido un v\'inculo entre el RI-UPPue y la ontolog\'ia Onto4RI-UPPue, lo que permite en primera instancia, evitar el retrabajo de insercci\'on de nuevos documentos por una lado y la inserci\'on de instancias por otro, al implementar un servicio de integraci\'on de instancias.\newline

Adem\'as, ya se cuenta con la posibilidad de exportar el contenido del RI-UPPue en formatos de intercambio de informaci\'on como lo son JSON y OWL, los cuales pueden ser manipulables por diferentes lenguajes de programaci\'on de alto nivel.\newline

Permite en una primera instancia establecer las relaciones existentes entre las personas que intervienen en la elaboraci\'on de una tesis de maestr\'ia, sin embargo, el servicio puede ser escalado para poder identificar cualquier tipo de relaciones existentes considerando otros criterios.\newline

La calidad de los datos almacenados en el RI-UPPue se vio incrementada al pasar de una estrella a cinco en la escala de los datos abiertos planteada por \cite{CincoEstrellas}, esto implica que inicialmente el RI-UPPue compart\'ia su informaci\'on en formato PDF evitando as\'i la posiblidad de la reutilizaci\'on, sin embargo, ahora se cuenta con el esquema de datos abiertos ligados.\newline

Como trabajo a futuro, se podr\'ia llevar a cabo la diseminaci\'on de los datos enriquecidos sem\'anticamente en formato OWL haciendo uso de protocolo de transmisi\'on de metadatos. \newline

Adem\'as se puede implementar un \textit{dashboard} para la toma de decisiones basadas en los indicadores generales originados por la producci\'on acad\'emica y cient\'ifica de la UPPue.\newline

Se puede escalar la expresividad de la ontolog\'ia al modelar mayor cantidad de clases, entidades y relaciones a trav\'es del an\'alisis e integraci\'on de mayor cantidad de metadatos extra\'idos a trav\'es del servicio web.\newline