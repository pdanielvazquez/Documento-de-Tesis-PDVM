\renewcommand\abstractname{Resumen}

\begin{abstract}
Un repositorio institucional (RI) es un espacio que almacena y preserva la producción
académica, científica y de innovación; apoya la gestión, preservación, catalogación, consulta e intercambio. Si bien algunos repositorios institucionales (RIs) cuentan con servicios tipo REST\footnote{\textit{Representational State Transfer}, transferencia del estado representacional}, el grado de complejidad en sus consultas considera un solo atributo, lo cual restringe al tipo de información que puede recuperarse.\newline

Este proyecto propone la implementación de un servicio web para recuperar información semántica del RI de la Universidad Politécnica de Puebla (RI-UPPue), los datos del repositorio relacionados con la organización de los contenidos, los usuarios y relaciones entre sí, están almacenados en una ontología, derivada de la descrita en \cite{representacionSemantica}. La ontología es una representación formal que puede ser procesada por la computadora y por las personas, implica una taxonomía, restricciones y reglas en los datos, permite además la validación automática de consistencia lógica.\newline

El servicio web propuesto pretende ser un punto de partida para explorar información semántica del RI-UPPue. Se considera extender las búsquedas que actualmente ofrece el RI-UPPue al implementar búsquedas avanzadas que utilicen dos o más metadatos, operadores lógicos, extracción de esquemas en formato RDF y JSON, unión de esquemas de metadatos que establezcan dominios comunes entre RIs y la posibilidad de gestionar datos abiertos enlazados (DAE) \cite{lodTheEssentials} .\newline

En el documento, se describe el contexto del servicio web propuesto y sus requerimientos desde la perspectiva de administradores o responsables, como en \cite{DSpaceRef} . Posteriormente, se utilizan herramientas de diseño UML como diagramas de clases y casos de uso para lograr obtener un diseño robusto, funcional y adaptado a las necesidades del software que se construirá en las fases subsecuentes. Se propone una arquitectura flexible que permita integrar otros servicios o soluciones similares. Finalmente, mediante el lenguaje de programación python se desarrolla el servicio web y se llevan a cabo sus pruebas de implementación, de lo cual derivan las conclusiones finales.\newline

\begin{keywords}
    Repositorio Institucional, RDF, JSON, OWL, Python, Servicio Web, DSpace, Ontología.
\end{keywords}

\end{abstract}